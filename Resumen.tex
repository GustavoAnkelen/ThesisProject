\textbf{M\'etricas de Einstein y Geometr\'ia de la Ecuaci\'on de Monge-Amp\`ere Compleja.}\\

Las ecuaciones de campo en relatividad general y sus soluciones son de inter\'es para estudiar y entender las interacciones gravitacionales como un fen\'omeno geom\'etrico del espacio-tiempo. Un caso particular, denominado como ecuaciones de campo del vac\'io dan lugar a una condici\'on de proporcionalidad entre la curvatura de Ricci de una m\'etrica y su forma fundamental sobre una variedad, esta condici\'on se denomina de Einstein. Esta memoria tratar\'a los t\'opicos de existencia de m\'etricas de K\"{a}hler que satisfacen la condici\'on de Einstein.

\vspace{2mm}

La condici\'on de proporcionalidad consiste en una igualdad en Cohomolog\'ia, esta igualdad puede utilizarse para deducir una ecuaci\'on en derivadas parciales no lineal del tipo Monge-Amp\`ere. 

\vspace{2mm}

Durante la segunda mitad del siglo pasado, se lograron variados y significativos avances en el problema de existencia de m\'etricas de K\"{a}hler-Einstein, especialmente por Yau y Aubin en los casos de curvatura negativa y nula. Adem\'as, Yau conjetur\'o que exist\'ia un obst\'aculo dado por alg\'un criterio de estabilidad cuando la curvatura define una clase de cohomolog\'ia positiva, conocida como primera clase de Chern. Este criterio de estabilidad fue establecido recientemente por Tian, que prob\'o, de forma paralela a Chen, Donaldson y Sun que cuando se verifica esta condici\'on de estabilidad, se satisface la existencia de m\'etricas de K\"{a}hler-Einstein.

\vspace{2mm}

En los siguientes 3 cap\'itulos la discusi\'on se enfoca en primer lugar en elementos b\'asicos de teor\'ia de Geometr\'ia compleja y Riemanniana, que son de gran importancia para la definici\'on y presentaci\'on posterior de las Variedades de K\"{a}hler. Luego, se presentan y desarrollan las tem\'aticas de espacios de H\"{o}lder y teor\'ia de Schauder que se utilizar\'an para introducir los conceptos de estimaciones a priori para soluciones de ecuaciones diferenciales. Finalmente, se presenta una prueba simplificada del Teorema de Yau (para los casos de curvatura negativa y nula) utilizando el m\'etodo de continuidad, as\'i como una aplicaci\'on adecuada del teorema de la funci\'on impl\'icita mediante el uso de las estimaciones a priori. La conclusi\'on de esta memoria consiste en la presentaci\'on de los obst\'aculos algebro-geom\'etricos para la existencia de m\'etricas de K\"{a}hler-Einstein cuando se tiene una primera clase de Chern positiva.

\vspace{2mm}

En el cap\'itulo final de esta memoria nos dedicamos a presentar consecuencias del teorema de Yau, o bien, de resultados cl\'asicos de variedades de K\"{a}hler-Einstein. De forma precisa, se presentar\'an de forma detallada: La desigualdad de Bogomolov-Miyaoka-Yau que entrega restricciones de gran utilidad para las clases de Chern (y por lo tanto, sobre la topolog\'ia) de estas variedades. Luego, el teorema de Descomposici\'on de Beauville-Bogomolov para variedades de K\"{a}hler Ricci planas, que nos dice (a groso modo) que los bloques fundamentales de este tipo de variedades son los toros complejos, las variedades de Calabi-Yau y las variedades Hiperk\"{a}hler. Finalmente, mostraremos que el teorema de Yau, junto con la aplicaci\'on de resultados cl\'asicos de geometr\'ia Riemanniana y compleja implican que toda variedad de Fano es simplemente conexa.
