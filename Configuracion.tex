\usepackage{amsmath}
\usepackage{amsfonts}
\usepackage{amssymb}
\usepackage{amsxtra}
%\usepackage[latin1]{inputenc}
\usepackage{epsfig}
\usepackage{graphicx}
\usepackage{enumerate}
\usepackage{float}
\usepackage{indentfirst}
\usepackage{graphicx}
\usepackage{rotating}
\usepackage{verbatim}
\usepackage{setspace}
\usepackage{cite}
\usepackage{color}
\usepackage{makeidx}
\usepackage[spanish]{babel}


%=================
% NUEVOS AMBIENTES
%=================

\newtheorem{nota}{\textbf{Nota}}
\newtheorem{proposicion}{\textbf{Proposici�n}}
\newtheorem{teorema}{\textbf{Teorema}}[chapter]
\newtheorem{ejemplo}{\textbf{Ejemplo}}[chapter]
\newtheorem{definicion}{\textbf{Definici�n}}[chapter]
\newtheorem{suposicion}{\textbf{Suposici�n}}
\newtheorem{procedimiento}{\textbf{Procedimiento}}


\def\pf{\noindent{\bf Demostraci�n}}
%==================
%FIN DEMOSTRACIONES
%==================

%\newcommand{\qed}{\begin{flushright}$\Box\Box\Box$\end{flushright}}

%==================
%FIN EJEMPLOS
%==================

\def\finejem{\vspace{5mm}\hspace*{\fill}~\mbox{\rule[0pt]{1.5ex}{1.5ex}}\par\endtrivlist\unskip}

%==================
%TEXTO EN COLOR
%==================

\newcommand {\fire}[1]{{\color[rgb]{0.6953, 0.1328, 0.1328} {#1}}}
\newcommand {\green}[1]{{\color[rgb]{0,0.3906,0} {#1}}}
\newcommand {\red}[1]{{\color[rgb]{1,0,0} {#1}}}
\newcommand {\blue}[1]{{\color[rgb]{0,0,0.5} {#1}}}

%==================
%CORTES DE PALABRAS
%==================

\hyphenation{con-ti-nua-cion cons-truc-cion ad-mi-si-ble
con-si-de-ra-ble res-trin-gi-da pro-ce-di-mien-to pro-ce-di-mien-tos
pro-ble-ma con-ti-nua-cion ad-mi-si-bles sen-si-bi-li-dad
es-ta-bi-li-dad pa-ra-me-tri-za-cion co-ro-la-rio fun-cio-nal
per-te-ne-cen fac-to-ri-za-cio-nes trans-fe-ren-cia rea-li-zar
co-pri-mas iz-quier-das ob-te-ne-mos pa-ra res-trin-gi-do u-san-do
mo-di-fi-ca-do des-pren-den cons-ti-tu-ye pro-ba-bi-li-dad
pro-ba-bi-li-da-des res-pues-ta va-lor ge-ne-ra-cion o-ri-gi-na-les
a-su-mien-do con-si-de-ra-das a-pro-xi-ma-cion res-tric-cio-nes
e-le-men-to va-ria-ble su-per-in-di-ce in-de-pen-dien-te e-di-fi-cio
des-pla-za-mien-to re-fe-ren-cia si-guien-te res-tric-cion
dis-po-si-ti-vos ni-vel res-tric-ti-vas e-le-men-tos ho-ri-zon-tal
va-ria-bles cons-tan-te cons-tan-tes con-fia-bi-li-dad res-tric-cion
fa-lla fa-llas gra-dien-te res-pe-tar}

%===================
%PAR�NTESIS Y LLAVES
%===================

\newcommand{\lpt}{\left.}
\newcommand{\rpt}{\right.}
\newcommand{\lc}{\left[}
\newcommand{\ly}{\left\{}
\newcommand{\ry}{\right\}}
\newcommand{\rc}{\right]}
\newcommand{\lp}{\left(}
\newcommand{\rp}{\right)}
\newcommand{\lb}{\left|}
\newcommand{\rb}{\right|}
\newcommand{\matrizsc}[1]{\begin{matrix}#1\end{matrix}}

%========
%S�MBOLOS
%========


\newcommand{\bs}[1]{\boldsymbol{#1}}
\newcommand{\PF}[1]{\mathbb{P}_{\emph{F}_i}(#1)}
\newcommand{\E}[1]{\mathbb{E}(#1)}




%===================
%OPERADORES
%===================
\newcommand{\traza}[1]{\ensuremath{\mathrm{traza}\left\{ #1 \right\}}}
\newcommand{\vectri}{\mathrm{vec_T}}
